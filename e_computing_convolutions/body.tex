\section{Convolution}

\begin{frame}{Introduction}
    \begin{enumerate}
        \item Using the convolution we can express the response of an LTI system to an arbitrary input in terms of the system's response to the unit impulse.
        \item An LTI system is completely characterized by its response to a single signal, namely, its response to the unit impulse.
        \item In discrete time, we have the convolution sum. In continuous time, we have the convolution integral.
        \item In this part of the course, we will learn to compute convolution sums and convolution integrals.
        \item In a latter section, under LTI system, we will obtain a thorough understanding of the concept of convolution.
    \end{enumerate}
\end{frame}

\begin{frame}{Convolution Sum}
    The convolution of the sequence $x[n]$ and $h[n]$ is given by
    \begin{equation}\label{eq:convolution_sum}
        y[n] = \sum_{k=-\infty}^{\infty}x[k]h[n-k],
    \end{equation}
    which we represent symbolically as
    \begin{equation}\label{eq:convolution_symbol}
        y[n] = x[n]\ast h[n]
    \end{equation}
\end{frame}

\begin{frame}{Example}
    Computer $y[n] = x[n]\ast h[n]$ for $x[n]$ and $h[n]$ as shown in Figure \ref{fi:example01_discrete_conv_signals}.
    \begin{figure}
      \centering
      \input{figures/example01_discrete_conv_signals}
      \caption{Computing convolution}\label{fi:example01_discrete_conv_signals}
    \end{figure}        
\end{frame}

\begin{frame}<beamer>[plain,t]
    \begin{figure}
      \centering
      \input{figures/example01_discrete_conv_01}
      \caption{Computing convolution}\label{fi:example01_discrete_conv_01}
    \end{figure}
\end{frame}

\begin{frame}<beamer>[plain,t]
    \begin{figure}
      \centering
      \input{figures/example01_discrete_conv_02}
      \caption{Computing convolution}\label{fi:example01_discrete_conv_02}
    \end{figure}
\end{frame}



\begin{frame}<beamer>[plain,t]
    \begin{figure}
      \centering
      \input{figures/example01_discrete_conv_03}
      \caption{Computing convolution}\label{fi:example01_discrete_conv_03}
    \end{figure}
\end{frame}




\begin{frame}{Example}
    Consider and input $x[n]$ and a unit impulse response $h[n]$ given by 
    \begin{equation}
        \begin{split}
            x[n] &= \alpha^nu[n]'\\
            h[n] &= u[n],
        \end{split}
    \end{equation}
    which $0 < \alpha < 1$. Find $y[n]$ and sketch.
\end{frame}

\begin{frame}
    \begin{enumerate}
        \item
    \end{enumerate}
\end{frame}



