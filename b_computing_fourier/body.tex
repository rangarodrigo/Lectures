\section{Computing Fourier Series and Transforms}

\subsection{Introduction}
\begin{frame}{Introduction}
    \begin{itemize}[<+->]
      \item Using the Fourier techniques we can obtain the frequency-domain representation of signals.
      \item We use Fourier series for periodic signals, and Fourier transform for aperiodic signals.
      \item Each of these have continuous-time and discrete-time versions:
        \begin{enumerate}
            \item Continuous-time Fourier series
            \item Continuous-time Fourier transform
            \item Discrete-time Fourier series
            \item Discrete-time Fourier transform
        \end{enumerate}
      \item In this part of the course, we will concentrate on how to actually compute these. Later, after we study liner, time-invariant (LTI) systems, we will study the conceptual aspects of Fourier techniques.
    \end{itemize}
\end{frame}

\begin{frame}
    \begin{columns}[t]
        \begin{column}{0.4\textwidth}
            \begin{figure}
              \centering
              \includegraphics[width=0.4\textwidth]{figures/fourier.jpg}
              \caption{Jean-Baptiste Joseph Fourier, 1768--1830, French mathematician who discovered Fourier series and transform}\label{fi:secb_fourier}
            \end{figure}
        \end{column}
        \begin{column}{0.6\textwidth}
            \begin{itemize}[<+->]
              \item Every signal has a frequency distribution or a \alert{spectrum}.
              \item Periodic signals have a line spectra, called the Fourier series.
              \item The French mathematician, Jean-Baptiste Joseph Fourier, discovered this representation.
              \item Fourier series provides a way  to represent a periodic signal as a sum of complex sinusoids.
              \item These sinusoids will be at frequencies that are integer multiples of the fundamental frequency $\omega_0$.
              \item $\omega_0 = \frac{2\pi}{T}$, where $T$: fundamental period of the waveform.
            \end{itemize}
        \end{column}
    \end{columns}
\end{frame}

\subsection{Fourier Series}

\begin{frame}{Continuous-Time Fourier Series}
    \begin{equation}\label{eq:secb_fourierseries}
        \begin{aligned}
            x(t) &= \sum_{k=-\infty}^{+\infty}a_k e^{jk\omega_0 t}\\
            a_k &= \frac{1}{T} \int_{T}x(t)e^{-jk\omega_0 t}dt\\
            \omega_0 &= \frac{2\pi}{T}
        \end{aligned}
    \end{equation}
    The set of coefficients $\left\{a_k\right\}$ is called the \alert{Fourier series coefficients} of the \alert{spectral coefficients} of $x(t)$.
    The coefficient $a_0$ is the dc or constant component of $x(t)$, given by Equation \ref{eq:secb_fourierseries} with $k=0$:
    \begin{equation}\label{eq:secb_a0}
        a_0 = \frac{1}{T} \int_{T}x(t)dt,
    \end{equation}
    which is simply the average of $x(t)$ over one period.
\end{frame}

\begin{frame}
    \begin{example}
        Let
        \begin{equation*}
            x(t) = 1 + \sin \omega_0t + 2\cos\omega_0t+ \cos\left(2\omega_0t+ \frac{\pi}{4}\right),
        \end{equation*}
        which has the fundamental frequency $\omega_0$.
        \begin{enumerate}
            \item Use Euler's formula to express $x(t)$ as a liner combination of complex exponentials.
            \item Find the Fourier series coefficients, $a_k$.
            \item Plot the magnitude and phase of $a_k$.
        \end{enumerate}

    \end{example}
\end{frame}

\begin{frame}<beamer>[plain,t]
        \begin{equation*}
            x(t) = 1 + \sin \omega_0t + 2\cos\omega_0t+ \cos\left(2\omega_0t+ \frac{\pi}{4}\right),
        \end{equation*}

        Using Euler's formula
        \begin{equation*}
            x(t) = 1 + \frac{1}{2j}\left[e^{j\omega_0 t} - e^{-j\omega_0 t}\right] + \left[e^{j\omega_0 t} + e^{-j\omega_0 t}\right] + \frac{1}{2}\left[e^{j(2\omega_0 t + \pi/4)} + e^{-j(2\omega_0 t + \pi/4)}\right]
        \end{equation*}
        Collecting terms,
        \begin{equation*}
            x(t) = 1 + \left(1 + \frac{1}{2j}\right)e^{j\omega_0 t} +  \left(1 - \frac{1}{2j}\right)e^{-j\omega_0 t} + \left(\frac{1}{2} e^{j\pi/4}\right)e^{j2\omega_0 t} + \left(\frac{1}{2} e^{-j\pi/4}\right)e^{-j2\omega_0 t}
        \end{equation*}
        \begin{columns}
            \begin{column}{0.5\textwidth}
                The Fourier coefficients are
                \begin{equation*}
                    \begin{split}
                        a_0 &= 1,\\
                        a_1 &= \left(1 + \frac{1}{2j}\right) = \left(1 - \frac{j}{2}\right),
                    \end{split}
                \end{equation*}
            \end{column}
            \begin{column}{0.5\textwidth}
                \begin{equation*}
                    \begin{split}
                        a_{-1} &= \left(1 + \frac{1}{2j}\right) = \left(1 + \frac{j}{2}\right),\\
                        a_2 &= \frac{1}{2}e^{j\pi/4} = \frac{\sqrt{2}}{4}(1+j),\\
                        a_{-2} &= \frac{1}{2}e^{-j\pi/4} = \frac{\sqrt{2}}{4}(1-j),\\
                        a_k &= 0, |k|>2.
                    \end{split}
                \end{equation*}
            \end{column}
        \end{columns}

\end{frame}

\begin{frame}<beamer>[plain,t]
    \begin{figure}
      \centering
      \input{figures/example01_fourier_euler}
      \caption{$|a_k|$, $\sphericalangle a_k$}\label{fi:secb-example01_fourier_euler}
    \end{figure}
\end{frame}



\begin{frame}
    \begin{example}
        The periodic square wave, sketched below, is defined over one period as
        \begin{equation*}
            x(t) = \begin{cases}
                1, & {t}<T_1,\\
                0, & T_1 < |t| < T/2,
            \end{cases}
        \end{equation*}
        This signal is periodic with fundamental period $T$ and fundamental frequency $\omega_0 = 2\pi/T$.
        \begin{enumerate}
            \item Find the Fourier series coefficients, $a_k$.
            \item Plot the magnitude and phase of $a_k$ for the case $T=4T_1$.
        \end{enumerate}
    \end{example}
\end{frame}

\begin{frame}<beamer>[plain,t]
    \begin{figure}
      \centering
      \input{figures/example02_periodic_square_wave }
      \caption{Periodic square wave}\label{fi:example02_periodic_square_wave }
    \end{figure}
\end{frame}

\begin{frame}<beamer>
    \begin{columns}
      \begin{column}{0.5\textwidth}
            \begin{equation*}
                \begin{split}
                   a_0 &= \frac{1}{T} \int_{T}x(t)dt,\\
                   &=  \frac{1}{T} \int_{-T_1}^{T_1}1dt,\\
                   &= \frac{2T_1}{T}.
                \end{split}
            \end{equation*}
            \pause
            \begin{equation*}
                \begin{split}
                   a_k &= \frac{1}{T} \int_{T}x(t)e^{-jk\omega_0 t}dt,\\
                   &=  \frac{1}{T} \int_{-T_1}^{T_1}e^{-jk\omega_0 t},\\
                   &= -\left. \frac{1}{jk\omega_0 T}e^{-jk\omega_0 t}\right|^{T_1}_{-T_1}
                \end{split}
            \end{equation*}
      \end{column}
      \begin{column}{0.5\textwidth}
            \pause
            \begin{equation*}
                \begin{split}
                   a_k &= \frac{2}{k\omega_0 T}\left[\frac{e^{jk\omega_0 t} - e^{-jk\omega_0 t}}{2j}\right]\\
                   a_k &= \frac{2\sin(k\omega_0 T_1)}{k\omega_0 T} = \frac{2\sin(k\omega_0 T_1)}{k\pi}, k \neq 0.
                \end{split}
            \end{equation*}
      \end{column}
    \end{columns}
\end{frame}

\begin{frame}<beamer>
    For $T=4T_1$
    \begin{equation*}
        a_k = 0, \quad k \text{ even}.
    \end{equation*}
    \begin{align*}
        a_1 &= a_{-1} = \frac{1}{\pi}\\
        a_3 &= a_{-3} = \frac{1}{3\pi}\\
        a_5 &= a_{-5} = \frac{1}{5\pi}\\
    \end{align*}
\end{frame} 


\begin{frame}<beamer>[plain,t]
    \begin{figure}
      \centering
      \input{figures/example02_periodic_square_fs}
      \caption{Plots of scaled Fourier series coefficients $Ta_k$}\label{fi:example02_periodic_square_fs}
    \end{figure}
\end{frame}
