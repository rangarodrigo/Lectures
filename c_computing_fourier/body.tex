\section{Continuous-Time Fourier Series}

\subsection{Introduction}
\begin{frame}{Introduction}
    \begin{itemize}[<+->]
      \item Using the Fourier techniques we can obtain the frequency-domain representation of signals.
      \item We use Fourier series for periodic signals, and Fourier transform for aperiodic signals.
      \item Each of these have continuous-time and discrete-time versions:
        \begin{enumerate}
            \item Continuous-time Fourier series
            \item Continuous-time Fourier transform
            \item Discrete-time Fourier series
            \item Discrete-time Fourier transform
        \end{enumerate}
      \item In this part of the course, we will concentrate on how to actually compute continuous-time Fourier series and transform. Later, after we study liner, time-invariant (LTI) systems, we will study the conceptual aspects of Fourier techniques.
    \end{itemize}
\end{frame}

\begin{frame}
    \begin{columns}[t]
        \begin{column}{0.4\textwidth}
            \begin{figure}
              \centering
              \includegraphics[width=0.4\textwidth]{figures/fourier.jpg}
              \caption{Jean-Baptiste Joseph Fourier, 1768--1830, French mathematician who discovered Fourier series and transform}\label{fi:secb_fourier}
            \end{figure}
        \end{column}
        \begin{column}{0.6\textwidth}
            \begin{itemize}[<+->]
              \item Every signal has a frequency distribution or a \alert{spectrum}.
              \item Periodic signals have a line spectra, called the Fourier series.
              \item The French mathematician, Jean-Baptiste Joseph Fourier, discovered this representation.
              \item Fourier series provides a way  to represent a periodic signal as a sum of complex sinusoids.
              \item These sinusoids will be at frequencies that are integer multiples of the fundamental frequency $\omega_0$.
              \item $\omega_0 = \frac{2\pi}{T}$, where $T$: fundamental period of the waveform.
            \end{itemize}
        \end{column}
    \end{columns}
\end{frame}

\subsection{Fourier Series}

\begin{frame}{Continuous-Time Fourier Series}
    \mode<beamer>
    {
        \begin{equation}\label{eq:secb_fourierseries}
            \begin{aligned}
                x(t) &= \sum_{k=-\infty}^{+\infty}a_k e^{jk\omega_0 t}\\
                a_k &= \frac{1}{T} \int_{T}x(t)e^{-jk\omega_0 t}dt\\
                \omega_0 &= \frac{2\pi}{T}
            \end{aligned}
        \end{equation}
        The set of coefficients $\left\{a_k\right\}$ is called the \alert{Fourier series coefficients} of the \alert{spectral coefficients} of $x(t)$.
        The coefficient $a_0$ is the dc or constant component of $x(t)$, given by Equation \ref{eq:secb_fourierseries} with $k=0$:
        \begin{equation}\label{eq:secb_a0}
            a_0 = \frac{1}{T} \int_{T}x(t)dt,
        \end{equation}
        which is simply the average of $x(t)$ over one period.
    }
\end{frame}

\begin{frame}[plain]
    \begin{example}
        Let
        \begin{equation*}
            x(t) = \sin \omega_0t,
        \end{equation*}
        which has the fundamental frequency $\omega_0$.
    \end{example}
    
    \mode<beamer>
    {
        \begin{equation*}
            \sin \omega_0t = \frac{1}{2j}e^{j\omega_0 t}- \frac{1}{2j}e^{-j\omega_0 t}
        \end{equation*}
        Comparing the right-hand side of this equation and Equation \ref{eq:secb_fourierseries}, we obtain
        \begin{equation*}
            \begin{split}
            a_1 &=  \frac{1}{2j} \qquad a_{-1} = -\frac{1}{2j}\\
            a_k &=0, \qquad k \neq \pm 1.\\            
            \end{split}
        \end{equation*}
    }
\end{frame}

\begin{frame}[plain]
    \begin{example}
        Let
        \begin{equation*}
            x(t) = 1 + \sin \omega_0t + 2\cos\omega_0t+ \cos\left(2\omega_0t+ \frac{\pi}{4}\right),
        \end{equation*}
        which has the fundamental frequency $\omega_0$.
        \begin{enumerate}
            \item Use Euler's formula to express $x(t)$ as a liner combination of complex exponentials.
            \item Find the Fourier series coefficients, $a_k$.
            \item Plot the magnitude and phase of $a_k$.
        \end{enumerate}

    \end{example}
\end{frame}

\begin{frame}<beamer>[plain,t]
        \begin{equation*}
            x(t) = 1 + \sin \omega_0t + 2\cos\omega_0t+ \cos\left(2\omega_0t+ \frac{\pi}{4}\right),
        \end{equation*}

        Using Euler's formula
        \begin{equation*}
            x(t) = 1 + \frac{1}{2j}\left[e^{j\omega_0 t} - e^{-j\omega_0 t}\right] + \left[e^{j\omega_0 t} + e^{-j\omega_0 t}\right] + \frac{1}{2}\left[e^{j(2\omega_0 t + \pi/4)} + e^{-j(2\omega_0 t + \pi/4)}\right]
        \end{equation*}
        Collecting terms,
        \begin{equation*}
            x(t) = 1 + \left(1 + \frac{1}{2j}\right)e^{j\omega_0 t} +  \left(1 - \frac{1}{2j}\right)e^{-j\omega_0 t} + \left(\frac{1}{2} e^{j\pi/4}\right)e^{j2\omega_0 t} + \left(\frac{1}{2} e^{-j\pi/4}\right)e^{-j2\omega_0 t}
        \end{equation*}
        \begin{columns}
            \begin{column}{0.5\textwidth}
                The Fourier coefficients are
                \begin{equation*}
                    \begin{split}
                        a_0 &= 1,\\
                        a_1 &= \left(1 + \frac{1}{2j}\right) = \left(1 - \frac{j}{2}\right),\\
                        a_{-1} &= \left(1 - \frac{1}{2j}\right) = \left(1 + \frac{j}{2}\right),\\
                    \end{split}
                \end{equation*}
            \end{column}
            \begin{column}{0.5\textwidth}
                \begin{equation*}
                    \begin{split}
                        a_2 &= \frac{1}{2}e^{j\pi/4} = \frac{\sqrt{2}}{4}(1+j),\\
                        a_{-2} &= \frac{1}{2}e^{-j\pi/4} = \frac{\sqrt{2}}{4}(1-j),\\
                        a_k &= 0, |k|>2.
                    \end{split}
                \end{equation*}
            \end{column}
        \end{columns}

\end{frame}

\begin{frame}<beamer>[plain,t]
    \begin{figure}
      \centering
      \input{figures/example01_fourier_euler}
      \caption{$|a_k|$, $\sphericalangle a_k$}\label{fi:secb-example01_fourier_euler}
    \end{figure}
\end{frame}



\begin{frame}
    \begin{example}
        The periodic square wave, sketched below, is defined over one period as
        \begin{equation*}
            x(t) = \begin{cases}
                1, & |t| <T_1,\\
                0, & T_1 < |t| < T/2,
            \end{cases}
        \end{equation*}
        This signal is periodic with fundamental period $T$ and fundamental frequency $\omega_0 = 2\pi/T$.
        \begin{enumerate}
            \item Find the Fourier series coefficients, $a_k$.
            \item Plot the magnitude and phase of $a_k$ for the case $T=4T_1$.
        \end{enumerate}
    \end{example}
\end{frame}

\begin{frame}[plain,t]
    \mode<beamer>
    {
        \begin{figure}
          \centering
          \input{figures/example02_periodic_square_wave }
          \caption{Periodic square wave}\label{fi:example02_periodic_square_wave }
        \end{figure}
    }
\end{frame}

\begin{frame}
    \mode<beamer>
    {
        \begin{columns}
          \begin{column}{0.5\textwidth}
                \begin{equation*}
                    \begin{split}
                       a_0 &= \frac{1}{T} \int_{T}x(t)dt,\\
                       &=  \frac{1}{T} \int_{-T_1}^{T_1}1dt,\\
                       &= \frac{2T_1}{T}.
                    \end{split}
                \end{equation*}
                \pause
                \begin{equation*}
                    \begin{split}
                       a_k &= \frac{1}{T} \int_{T}x(t)e^{-jk\omega_0 t}dt,\\
                       &=  \frac{1}{T} \int_{-T_1}^{T_1}e^{-jk\omega_0 t},\\
                       &= -\left. \frac{1}{jk\omega_0 T}e^{-jk\omega_0 t}\right|^{T_1}_{-T_1}
                    \end{split}
                \end{equation*}
          \end{column}
          \begin{column}{0.5\textwidth}
                \pause
                \begin{equation*}
                    \begin{split}
                       a_k &= \frac{2}{k\omega_0 T}\left[\frac{e^{jk\omega_0 t} - e^{-jk\omega_0 t}}{2j}\right]\\
                       a_k &= \frac{2\sin(k\omega_0 T_1)}{k\omega_0 T} = \frac{2\sin(k\omega_0 T_1)}{k\pi}, k \neq 0.
                    \end{split}
                \end{equation*}
          \end{column}
        \end{columns}
    }
\end{frame}

\begin{frame}
    \mode<beamer>
    {
    For $T=4T_1$
    \begin{equation*}
        a_k = 0, \quad k \text{ even}.
    \end{equation*}
    \begin{align*}
        a_0 &= \frac{1}{2}\\
        a_1 &= a_{-1} = \frac{1}{\pi}\\
        a_3 &= a_{-3} = \frac{1}{3\pi}\\
        a_5 &= a_{-5} = \frac{1}{5\pi}\\
    \end{align*}
    }
\end{frame}


\begin{frame}<beamer>[plain,t]
    \begin{figure}
      \centering
      \input{figures/example02_periodic_square_fs}
      \caption{Plots of scaled Fourier series coefficients $Ta_k$}\label{fi:example02_periodic_square_fs}
    \end{figure}
\end{frame}


\section{Properties of the Continuous-Time Fourier Series}

\begin{frame}
    Suppose that $x(t)$ is a periodic signal with period $T$ and fundamental frequency $\omega_0 = 2\pi/T$. Then if the Fourier series coefficients are denoted by $a_k$, then
    \begin{equation}
        x(t) \overset{\mathcal{FS}}{\longleftrightarrow} a_k
    \end{equation}
\end{frame}

\begin{frame}{Linearity}
    Let $x(t)$ and $y(t)$ denote two periodic signals with period $T$.
    \begin{align*}
        x(t) &\overset{\mathcal{FS}}{\longleftrightarrow} a_k,\\
        y(t) &\overset{\mathcal{FS}}{\longleftrightarrow} b_k.\\
    \end{align*}
    Any linear combination of the two signals will also be periodic with period $T$. Fourier series coefficients $c_k$ of the linear combination of $X(t)$ and $y(t)$, $z(t) = Ax(t) + By(t)$, are given by the same linear combination:
    \mode<beamer>
    {
        \begin{equation}
            z(t) = Ax(t) + By(t)\overset{\mathcal{FS}}{\longleftrightarrow} c_k = Aa_k + Bb_k.
        \end{equation}
    }
\end{frame}

\begin{frame}[plain]{Time Shifting}
    \begin{equation}
        x(t-t_0) \overset{\mathcal{FS}}{\longleftrightarrow} e^{-jk\omega_0 t_0}a_k = e^{-jk(2\pi/T) t_0}a_k
    \end{equation}
    \mode<beamer>
    {
        \noindent Proof:\\%Page 203
        \pause
        \begin{equation*}
            \begin{aligned}
          x(t) &\overset{\mathcal{FS}}{\longleftrightarrow} a_k,\\            
            x(t-t_0) &\overset{\mathcal{FS}}{\longleftrightarrow} b_k,\\
                b_k &=  \frac{1}{T}\int_{T}x(t-t_0)e^{-jk\omega_0 (\tau + t_0)}d\tau,\\
                &= e^{-jk\omega_0  t_0}\frac{1}{T}\int_{T}x(t-t_0)e^{-jk\omega_0\tau)}d\tau,\\
                &= e^{-jk\omega_0  t_0}a_k.
            \end{aligned}
        \end{equation*}
        
        \begin{equation*}
             x(t-t_0) \overset{\mathcal{FS}}{\longleftrightarrow} e^{-jk\omega_0  t_0}a_k.
        \end{equation*}

        Note: $|a_k| = |b_k|$
    }
\end{frame}


\begin{frame}{Time Reversal}
    If
    \begin{equation}
        x(t) \overset{\mathcal{FS}}{\longleftrightarrow} a_k
    \end{equation}
    then
    \begin{equation}
        x(-t) \overset{\mathcal{FS}}{\longleftrightarrow} a_{-k}.
    \end{equation}
    \mode<beamer>
    {
        \begin{equation}
            x(-t) =  \sum_{k=-\infty}^{\infty}a_ke^{-jk2\pi t/T}.
        \end{equation}
        Substitution: $k=-m$
        \begin{equation*}
            y(t) = x(-t) = \sum_{m=-\infty}^{\infty}a_{-m}e^{-jk2\pi t/T}.
        \end{equation*}
        \pause
        
    }
\end{frame}

\begin{frame}<beamer>
    \begin{itemize}[<+->]
      \item Time reversal applied to a continuous-time signal results in a time reversal of the corresponding sequence of Fourier series coefficients.
      \item If $x(t)$ is even---$x(-t) = x(t)$---then its Fourier series coefficients are also even, i.e., $a_{-k}=a_k$.
      \item If $x(t)$ is odd---$x(-t) = -x(t)$---then its Fourier series coefficients are also odd, i.e., $a_{-k}=-a_k$.
    \end{itemize}
\end{frame}


\begin{frame}{Time Scaling}
    \mode<beamer>
    {
        Time scaling, in general, changes the period.\\
        If $x(t)$ is a periodic with period $T$ and fundamental frequency $\omega_0 = 2\pi/T$, then $x(\alpha t)$, where $\alpha$ is a positive real number, is periodic with period $T/\alpha$ and fundamental frequency $\alpha \omega_0$.
        \begin{equation}
            x(\alpha T) = \sum_{k=-\infty}^{\infty} a_k e^{jk(\alpha \omega_0)t}
        \end{equation}
        While Fourier coefficients have not changes, the Fourier series representation \alert{has} changed because of the change in the fundamental frequency.

    }
\end{frame}


\begin{frame}{Multiplication}
    \mode<beamer>
    {
        Let $x(t)$ and $y(t)$ denote two periodic signals with period $T$.
        \begin{align*}
            x(t) &\overset{\mathcal{FS}}{\longleftrightarrow} a_k,\\
            y(t) &\overset{\mathcal{FS}}{\longleftrightarrow} b_k.\\
        \end{align*}
        Since the product $x(t)y(t)$ is also periodic with period $T$, its Fourier series coefficients $h_k$ are
        \begin{equation}
            x(t)y(t) \overset{\mathcal{FS}}{\longleftrightarrow} \sum_{l=-\infty}^{\infty}a_l b_{k-l}. 
        \end{equation}

    }
\end{frame}


\begin{frame}{Conjugation and Conjugate Symmetry}
    \mode<beamer>
    {
        
        \begin{itemize}[<+->]
          \item Taking the complex conjugate of a periodic signal$x(t)$ has the effect of complex conjugation and \alert{time reversal} on the corresponding Fourier series coefficients. 
          \item If $x(t)$ is real---$x(t) = x^\ast(t)$: Fourier series coefficients conjugate symmetric,i.e., $a_{-k} = a^\ast_k$.
          \item If $x(t)$ is real, then $a_0$ is real and $|a_k| = |a_{-k}|$.
          \item If $x(t)$ is real and even, we know that $a_k = a_{-k}$. From above, $a^\ast_k = a_{-k}$, so that $a_{k} = a^\ast_k$. That is if $x(t)$ is real and even, so are its Fourier series coefficients.
          \item If $x(t)$ is real and odd, its Fourier series coefficients are purely imaginary and odd. Thus, e.g., $a_0 = 0$.% do probelm 3.42 in tutorial.
        \end{itemize}
    }
\end{frame}


\begin{frame}{Parseval's Relation for Continuous-Time Periodic Signals}
    \begin{equation}\label{eq:parseval}
        \frac{1}{T}\int_{T} |x(t)|^2dt = \sum_{k=-\infty}^{\infty}|a_k|^2.
    \end{equation}
    \mode<beamer>
    {
        Note: Left-hand side of equation \ref{eq:parseval} is the average power (i.e., energy per unit time) in one period of the periodic signal $x(t)$.\\

        \begin{equation}
            \frac{1}{T}\int_{T} \left| a_k e^{jk\omega_0 t}\right|^2dt = \frac{1}{T}\int_{T} \left| a_k \right|^2dt = |a_k|^2.
        \end{equation}
        So, $|a_k|^2$ is the average power in the $k$th harmonic component of $x(k)$.\\
        Thus, what Parseval's relation state is that the total power in a periodic signal equals the sum of the average powers in all of its harmonic components.
    }
\end{frame}

\begin{frame}
    \begin{example}%3.6
        Consider the signal $g(t)$ with a fundamental period of $4$, shown in Figure \ref{fi:example3p6}.
        \begin{figure}
          \centering
          \input{figures/example3p6}
          \caption{Figure for example}\label{fi:example3p6}
        \end{figure}

        Determine the Fourier series representation of $g(t)$
        \begin{enumerate}
            \item directly from the analysis equation.
            \item by assuming that the Fourier series coefficients of the symmetric periodic square wave are known.
        \end{enumerate}
    \end{example}
\end{frame}

\begin{frame}
\mode<beamer>
{
}
\end{frame}



\begin{frame}%3.7
    \begin{example}
        Consider the triangular wave signal $x(t)$ with period $T=4$ and fundamental frequency $\omega_0 = \pi/2$, shown in Figure \ref{fi:example3p7}. The derivative signal is the signal $g(t)$ in Figure \ref{fi:example3p6}. Using this information, find the Fourier series coefficients of $x(t)$.
        \begin{figure}
          \centering
          \input{figures/example3p7}
          \caption{Figure for example}\label{fi:example3p7}
        \end{figure}
    \end{example}
\end{frame}

\begin{frame}
\mode<beamer>
{
}
\end{frame}


\begin{frame}%3.8
    \begin{example}
        Obtain the Fourier series coefficients of the impulse train
        \begin{equation}\label{eq:impulsetrain}
            x(t) = \sum_{k=-\infty}^{\infty} \delta(t-kT).
        \end{equation}
    \end{example}
\end{frame}

\begin{frame}
\mode<beamer>
{
}
\end{frame}


\begin{frame}
    \begin{example}
        By expressing the derivative of a square wave signal in terms of impulses, obtain the Fourier series coefficients of the square wave signal.
        \begin{figure}
          \centering
          \input{figures/example3p8_square}
          \caption{Figure for example}\label{fi:example3p8_square}
        \end{figure}
    \end{example}
\end{frame}

\begin{frame}
\mode<beamer>
{
}
\end{frame}
